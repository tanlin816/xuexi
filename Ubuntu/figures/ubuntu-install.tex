\documentclass[12pt]{article}
%%---------------------------------------------------------------------
% packages
% geometry
\usepackage{geometry}
% font
\usepackage{fontspec}
\defaultfontfeatures{Mapping=tex-text}  %%如果没有它,会有一些 tex 特殊字符无法正常使用,比如连字符。
\usepackage{xunicode,xltxtra}
\usepackage[BoldFont,SlantFont,CJKnumber,CJKchecksingle]{xeCJK}  % \CJKnumber{12345}: 一万二千三百四十五
\usepackage{CJKfntef}  %%实现对汉字加点、下划线等。
\usepackage{pifont}  % \ding{}
% math
\usepackage{amsmath,amsfonts,amssymb}
% color
\usepackage{color}
\usepackage{xcolor}
\definecolor{YELLOW}{RGB}{255,255,224}
\definecolor{EYE}{RGB}{199,237,204}
\definecolor{FLY}{RGB}{128,0,128}
\definecolor{ZHY}{RGB}{139,0,255}
% graphics
\usepackage[americaninductors,europeanresistors]{circuitikz}
\usepackage{tikz}
\usetikzlibrary{positioning,arrows,shadows,shapes,calc,mindmap,trees,backgrounds}  % placements=positioning
\usepackage{graphicx}  % \includegraphics[]{}
\usepackage{subfigure}  %%图形或表格并排排列
% table
\usepackage{colortbl,dcolumn}  %% 彩色表格
\usepackage{multirow}
\usepackage{multicol}
\usepackage{booktabs}
% code
\usepackage{fancyvrb}
\usepackage{listings}
% title
%\usepackage{titlesec}
% head/foot
\usepackage{fancyhdr}
% ref
\usepackage{hyperref}
% pagecolor
\usepackage[pagecolor={YELLOW}]{pagecolor}
% tightly-packed lists
\usepackage{mdwlist}

\usepackage{styles/iplouccfg}
\usepackage{styles/zhfontcfg}
\usepackage{styles/iplouclistings}
\usepackage{enumitem}
%%---------------------------------------------------------------------
% settings
% geometry
\geometry{left=2cm,right=1cm,top=2cm,bottom=2cm}  %设置 上、左、下、右 页边距
\linespread{1.5} %行间距
% font
\setCJKmainfont{Adobe Kaiti Std}
%\setmainfont[BoldFont=Adobe Garamond Pro Bold]{Apple Garamond}  % 英文字体
%\setmainfont[BoldFont=Adobe Garamond Pro Bold,SmallCapsFont=Apple Garamond,SmallCapsFeatures={Scale=0.7}]{Apple Garamond}  %%苹果字体没有SmallCaps
\setCJKmonofont{Adobe Fangsong Std}
% graphics
\graphicspath{{figures/}}
\tikzset{
    % Define standard arrow tip
    >=stealth',
    % Define style for boxes
    punkt/.style={
           rectangle,
           rounded corners,
           draw=black, very thick,
           text width=6.5em,
           minimum height=2em,
           text centered},
    % Define arrow style
    pil/.style={
           ->,
           thick,
           shorten <=2pt,
           shorten >=2pt,},
    % Define style for FlyZhyBall
    FlyZhyBall/.style={
      circle,
      minimum size=6mm,
      inner sep=0.5pt,
      ball color=red!50!blue,
      text=white,},
    % Define style for FlyZhyRectangle
    FlyZhyRectangle/.style={
      rectangle,
      rounded corners,
      minimum size=6mm,
      ball color=red!50!blue,
      text=white,},
    % Define style for zhyfly
    zhyfly/.style={
      rectangle,
      rounded corners,
      minimum size=6mm,
      ball color=red!25!blue,
      text=white,},
    % Define style for new rectangle
    nrectangle/.style={
      rectangle,
      draw=#1!50,
      fill=#1!20,
      minimum size=5mm,
      inner sep=0.1pt,}
}
\ctikzset{
  bipoles/length=.8cm
}
% code
\lstnewenvironment{VHDLcode}[1][]{%
  \lstset{
    basicstyle=\footnotesize\ttfamily\color{black},%
    columns=flexible,%
    framexleftmargin=.7mm,frame=shadowbox,%
    rulesepcolor=\color{blue},%
%    frame=single,%
    backgroundcolor=\color{yellow!20},%
    xleftmargin=1.2\fboxsep,%
    xrightmargin=.7\fboxsep,%
    numbers=left,numberstyle=\tiny\color{blue},%
    numberblanklines=false,numbersep=7pt,%
    language=VHDL%
    }\lstset{#1}}{}
\lstnewenvironment{VHDLmiddle}[1][]{%
  \lstset{
    basicstyle=\scriptsize\ttfamily\color{black},%
    columns=flexible,%
    framexleftmargin=.7mm,frame=shadowbox,%
    rulesepcolor=\color{blue},%
%    frame=single,%
    backgroundcolor=\color{yellow!20},%
    xleftmargin=1.2\fboxsep,%
    xrightmargin=.7\fboxsep,%
    numbers=left,numberstyle=\tiny\color{blue},%
    numberblanklines=false,numbersep=7pt,%
    language=VHDL%
    }\lstset{#1}}{}
\lstnewenvironment{VHDLsmall}[1][]{%
  \lstset{
    basicstyle=\tiny\ttfamily\color{black},%
    columns=flexible,%
    framexleftmargin=.7mm,frame=shadowbox,%
    rulesepcolor=\color{blue},%
%    frame=single,%
    backgroundcolor=\color{yellow!20},%
    xleftmargin=1.2\fboxsep,%
    xrightmargin=.7\fboxsep,%
    numbers=left,numberstyle=\tiny\color{blue},%
    numberblanklines=false,numbersep=7pt,%
    language=VHDL%
    }\lstset{#1}}{}
% pdf
\hypersetup{pdfpagemode=FullScreen,%
            pdfauthor={Haiyong Zheng},%
            pdftitle={Title},%
            CJKbookmarks=true,%
            bookmarksnumbered=true,%
            bookmarksopen=false,%
            plainpages=false,%
            colorlinks=true,%
            citecolor=green,%
            filecolor=magenta,%
            linkcolor=cyan,%red(default)
            urlcolor=cyan}
% section
%http://tex.stackexchange.com/questions/34288/how-to-place-a-shaded-box-around-a-section-label-and-name
\newcommand\titlebar{%
\tikz[baseline,trim left=3.1cm,trim right=3cm] {
    \fill [cyan!25] (2.5cm,-1ex) rectangle (\textwidth+3.1cm,2.5ex);
    \node [
        fill=cyan!60!white,
        anchor= base east,
        rounded rectangle,
        minimum height=3.5ex] at (3cm,0) {
        \textbf{\thesection.}
    };
}%
}
%\titleformat{\section}{\Large\bf\color{blue}}{\titlebar}{0.1cm}{}
% head/foot
\setlength{\headheight}{15pt}
\pagestyle{fancy}
\fancyhf{}
\chead{\color{red}UBUTU'S INSTALL AND SIMPLE USE}%页眉
\cfoot{\color{red}July2016}%页脚 中
\lfoot{\color{red}谭琳\ 丁昊\ 崔金娜}%页脚 左
\rfoot{\color{red}$\cdot$\ \thepage\ $\cdot$}%页脚 右
\renewcommand{\headrulewidth}{0.4pt}
\renewcommand{\footrulewidth}{0.4pt}
%%---------------------------------------------------------------------
\begin{document}
\title{\vspace{-2em}Ubuntu 16.04安装教程及简单运用\vspace{0.7em}}%大标题
\author{谭琳\ 丁昊\ 崔金娜}%作者
\date{\vspace{-0.7em}2016年7月\vspace{-0.7em}}%日期
\maketitle\thispagestyle{fancy}%在上部添加横线
\maketitle
\tableofcontents 
\section{ubuntu的安装}

现在实验室用的是ubuntu16.04,至于版本根据ubuntu官方出的最新版为准,因为ubuntu的安装近乎相同,这里以ubuntu16.04的安装为例进行说明。

\subsection{ubuntu的下载}

  在ubuntu官网上下载http://cn.ubuntu.com/download/,如果你电脑的内存少于2GB,选择32位下载,我选择下载的的是64位。下载下来是一个ubuntu-16.04-desktop-amd64.iso的镜像文件,这个下载过程有点长,需要你耐心等待哦!

\subsection{u盘启动器的制作}

  在安装ubuntu之前需要做一个硬件启动器,你可以用easyBCD软件来安装,也可以用u盘来制作u盘启动器,二者我都用过,觉得还是u盘比较简单,这里我就以制作u盘为例。

  首先找个最好是8G的空u盘,插在你的电脑上备用。然后在百度上可以搜软碟通ultraiso,这个在windows7和windows8都可以下载。下载完成后打开ultraiso,点击工具栏的“打开”按钮,选择第一步下载的ubuntu-16.04-desktop-amd64.iso打开。

  再点击工具栏里的“启动”,在弹出的框里,点击“写入硬盘映像”,如图\ref{tu1}所示。
\begin{figure}[!htb] %插图
\centering
\includegraphics[width=0.6\textwidth]{tu1.jpeg}
\caption{\small}
\label{tu1}
\end{figure}  

  此时,在弹出的框内,硬盘驱动器是你的u盘名,写入方式是USB-DD。下面一步是我们制作的u盘启动器能否启动电脑的关键,点击“便捷启动”,选择“写入新的驱动器引导扇面去”的“Syslinux”。随后,会弹出很多提示框,你可以一路点是到底,这一步的Syslinux写入很快的,最后会提示你安装成功,如若不然,你可重复上面的步骤。如图\ref{tu2}图\ref{tu3}所示。
\begin{figure}[!htb] %插图
\centering
\includegraphics[width=0.6\textwidth]{tu2.jpeg}
\caption{}
\label{tu2}
\end{figure}  

\begin{figure}[!htb] %插图
\centering
\includegraphics[width=0.6\textwidth]{tu3.jpeg}
\caption{}
\label{tu3}
\end{figure}  
  制作u盘启动器的最后一步就是,点击“写入”,大约此过程需要等待写入完成5分钟,随后,在消息里可以看到“刻录成功”,我们点击“返回”即可,如图\ref{tu5}所示。

\begin{figure}[!htb] %插图
\centering
\includegraphics[width=0.6\textwidth]{tu5.jpeg}
\caption{}
\label{tu5}
\end{figure} 
 
此时,我们的u盘启动器就制作好了,我们可以安装我们的ubuntu啦!

\subsection{ubuntu的安装}

\subsubsection{ubuntu单系统安装}

  1 刚刚安装好的u盘启动器不用拔出,直接重启电脑,一直按快捷键进入boot界面。各款电脑常用快捷键有:笔记本 联想、宏基、三星、方正、海尔、清华同方、戴尔、神舟 F12,华硕、索尼 ESC,惠普、明基 F9;台式机 联想、惠普、宏基、神舟、方正、海尔、清华同方 F12,戴尔 ESC,华硕、明基 F8;

  2 进入boot界面后,选择“USB”那个栏安装,如下图\ref{tu6}所示(因为我按的就是ubuntu,所以有好几个USB选项,第一次装的话,那么就一个)。

\begin{figure}[!htb] %插图
\centering
\includegraphics[width=0.5\textwidth]{tu6.jpeg}
\caption{}
\label{tu6}
\end{figure} 

  3 进入之后,在左侧选择“中文(简体)”,那么右侧就出现了如下图\ref{tu7}所示的安装界面,点击“安装ubuntu”。

\begin{figure}[!htb] %插图
\centering
\includegraphics[width=0.5\textwidth]{tu7.jpeg}
\caption{}
\label{tu7}
\end{figure}
 
  4 在出现的提示框下\ref{tu8},一般是不选择,直接按“继续”。

\begin{figure}[!htb] %插图
\centering
\includegraphics[width=0.5\textwidth]{tu8.jpeg}
\caption{}
\label{tu8}
\end{figure} 

  5 随后出现安装类型,如下图\ref{tu9},(因为我已经安装了ubuntu16.04),如果你原先是windows7,你想保留windows7的一些文档、音乐、其他文件可以选择第二个与你原先的并存。在实验室,一般都是单系统,所以你也以选择第三项清除整个磁盘,但是这样在运行一些程序有点卡。所以一般还是选择按“其他选项”,这样可以自己分区,如果你想安装双系统的话,一定的选{\color{red}“其他选项”},至于安装双系统教程下面会说。点击“继续”。

\begin{figure}[!htb] %插图
\centering
\includegraphics[width=0.5\textwidth]{tu9.jpeg}
\caption{}
\label{tu9}
\end{figure} 
\begin{figure}[!htb] %插图
\centering
\includegraphics[width=0.5\textwidth]{tu10.jpeg}
\caption{}
\label{tu10}
\end{figure} 

  6 进入分区界面\ref{tu10}。单击左下角“-”是删除分区,“+”是创建分区。ubuntu分区一般有/boot、/swap、/、/home、/opt五个挂载点。实验室一般安装的都是单系统,这里以安装单系统内存为8G,硬盘空间500G为例,说一下分区。

分区类型选的 都是“逻辑分区”,新分区的位置选择,是“空间起始位置”。

/boot,单系统不用分

/    ,分大约65G就行 ,用于Ext4日志文件系统

/home,分大约250G就行,用于Ext4日志文件系统

/swap,内存在4G以上可以不用考虑,如果要分的话,分大约是内存的1.5-2倍,用于交换空间

/opt ,余下的都分在这个区,用于用于Ext4日志文件系统

  7 分区分好之后选择“现在安装”。

  8 出现您在什么地方的提示框,你只要单击“继续”就可以了。图\ref{tu12}所示

\begin{figure}[!htb] %插图
\centering
\includegraphics[width=0.4\textwidth]{tu12.jpeg}
\caption{}
\label{tu12}
\end{figure} 

  9 在键盘布局选择英语(美国),点击“继续”。图\ref{tu13}所示
\begin{figure}[!htb] %插图
\centering
\includegraphics[width=0.4\textwidth]{tu13.jpeg}
\caption{}
\label{tu13}
\end{figure} 

  10填写个人信息,名字密码,单击“继续”。

  11进入到了安装界面,等待安装

\begin{figure}[!htb] %插图
\centering
\includegraphics[width=0.5\textwidth]{tu14.jpeg}
\caption{}
\label{tu14}
\end{figure} 

安装完成后,你需要的是重启电脑,图\ref{tu14}所示。开机画面是这样的,你需要输入刚才设置的密码即可。此时你的ubuntu就安装上了你可以使用了哦!

\subsubsection{ubuntu双系统安装}

如果你要装双系统的话,{\color{red}温馨提示一下},{\color{blue}如果新手的话,操作难免会失误,因此,建议大家操作前,切记注意备份重要数据,本人就是一个血淋淋的例子}。

 1 双系统也需要下载,制作u盘启动器,在安装之前需要在分区这里用windows7自带的磁盘管理进行分区,打开方法:右键计算机-管理-磁盘管理。如图\ref{tu21}所示
\begin{figure}[!htb] %插图
\centering
\includegraphics[width=0.2\textwidth]{tu21.jpeg}
\caption{}
\label{tu21}
\end{figure} 

 2 找一个你电脑上比较大的磁盘就可以啦。方法:对着要压缩的分区右击选择“压缩卷即可”,我分了100G,也可以不用这么大,但记得千万别格式化,也别分配盘符,压缩完就可以了。

 3 开始安装与上面一样,就是在安装类型时选择“其他选项”,点击“继续”。

 4 找到其中标有“空闲”的盘符,这个盘符就是我们用于安装ubuntu的100G空间,别去碰别的盘符,小心弄得到时候win7不能用了,甚至品牌机自带的隐藏分区也会被破坏。图\ref{tu23}所示
\begin{figure}[!htb] %插图
\centering
\includegraphics[width=0.4\textwidth]{tu23.jpeg}
\caption{}
\label{tu23}
\end{figure} 

5 点击左下方的“+”,挂载点:/,大小:22000MB ,新分区的类型:主分区,新分区的位置:空间起始位置,用于:EXT4日志文件系统;继续“+”,挂载点:/boot,大小:200MB(一般分100-200MB,但是你的硬盘够大,分1-2G比较好使),新分区的类型:逻辑分区,新分区的位置:空间起始位置,用于:EXT4日志文件系统;挂载点:/home,大小50G:新分区的类型:逻辑分区,新分区的位置:空间起始位置,用于:EXT4日志文件系统;/opt、/swap分区与上面相似,这里就不说了。图\ref{tu24}所示
\begin{figure}[!htb] %插图
\centering
\includegraphics[width=0.3\textwidth]{tu24.jpeg}
\caption{}
\label{tu24}
\end{figure} 
\begin{figure}[!htb] %插图
\centering
\includegraphics[width=0.3\textwidth]{tu25.jpeg}
\caption{}
\label{tu25}
\end{figure} 

6 下面步骤与单系统安装一样,就是安装完之后开机界面是这样的\ref{tu25},你可以上下选择进入windows还是ubuntu。

到目前为止,ubuntu就安装好了,你可以正常使用了!
\section{ubuntu的简单使用}
 安装好ubuntu,接下来就是使用了
\subsection{ubuntu升级}
  刚按上ubuntu,需要更新一下,否则许多软件包无法正常使用。

 1、打开终端:Ctrl+Alt+T 。为了下次方便,打开之后可以在左侧的终端符号除右击选择“锁定到启动器”。如图\ref{tu32}所示

\begin{figure}[!htb] %插图
\centering
\includegraphics[width=0.3\textwidth]{tu32.jpeg}
\caption{}
\label{tu32}
\end{figure} 

 2、在终端输入: sudo apt-get update

~~~~~~~~~~~~~~~~~~~~~~~sudo apt-get upgrade\\
{\color{red}注意:}ubuntu下使用{\color{red}“sudo”}是以管理员权限运行命令的!
\subsection{ubuntu常用命令}
\subsubsection{ls -al}
 ls是“list”的意思,它主要是显示文件的文件名与相关属性。“-al”表示列出所有的文件详细的权限与属性(包含第一字符为“.”的隐藏文件)。我以我的电脑为例,在终端输入以上命令,可以出现如图\ref{tu31}
\begin{figure}[!htb] %插图
\centering
\includegraphics[width=0.4\textwidth]{tu31.jpeg}
\caption{}
\label{tu31}
\end{figure} 
\begin{itemize}

\item 第一列代表这个文件的类型与权限

第一个字符是d代表是目录

第一个字符是-代表是文件

第一个字符是l代表是链接文件
\item[*] 接下来3个字符为一组,一共三组。每一组的r是可读,w是可写,x是可执行。这三个权限位置循序是不变的,如果没有权限,就用“-”。这三组权限分别代表:

第一组,文件所有者的权限

第二组,同用户组的权限

第三组,其他非本用户组的权限

\item  第二列代表有多少文件名连接到此节点

\item  第三列代表这个文件的所有者账号

\item  第四列代表这个文件的所属用户组

\item  第五列代表这个文件的容量的大小,默认单位为B

\item  第六列代表这个文件的创建文件日期或者最近的修改时间

\item  第七列代表该文件名
这七个字段的意义重要。尤其第一个字段的9个权限是linux的重点

\end{itemize}

%第一列代表这个文件的类型与权限
 %第一个字符是d代表是目录
% 第一个字符是-代表是文件
% 第一个字符是l代表是链接文件
% 接下来3个字符为一组,一共三组。每一组的r是可读,w是可写,x是可执行。这三个权限位置循序是不变的,如果没有权限,就用“-”。这三组权限分别代表:
 %第一组,文件所有者的权限
% 第二组,同用户组的权限
 %第三组,其他非本用户组的权限
%第二列代表有多少文件名连接到此节点
%第三列代表这个文件的所有者账号
%第四列代表这个文件的所属用户组
%第五列代表这个文件的容量的大小,默认单位为B
%第六列代表这个文件的创建文件日期或者最近的修改时间
%第七列代表该文件名
%这七个字段的意义重要。尤其第一个字段的9个权限是linux的重点
\subsubsection{cd 切换目录命令}
“cd ..”代表回到上一层目录。“cd /”代表回到根目录。“cd ~”代表回到自己的主文件下即你自己用户名下。“cd -”代表回到上一步目录下。“cd”代表回到自己的主文件下。“cd /home/用户名/想要去的目录/想要去的文件或者子目录”可以从根目录跳转到你想要去的目录或者文件下。你在执行这些命令时,可以用“ls”查看一下执行这一步后都有哪些目录或文件,很容易就理解了。
\subsubsection{pwd}

“pwd”显示目前所在的目录,当前的一个路径
\subsubsection{mkdir}
“mkdir 新的目录名”表示新建新目录。

新建目录是一层层依次建的,即用cd命令依次进入目录里面创建。不过,也可以在进入你想要建总的目录下用“mkdir -p 目录1/目录2/目录3 ”,可以建立三个依次三个子目录。
\subsubsection{rmdir}
“rmdir 目录名”可以删除空的目录。

也是cd,进入目录,使用“rmdir”就可以删除刚刚新建的目录,你可以一层层进入删除,可以以用“rkdir -p 目录1/目录2/目录3”,一次性删除。
{\color{red}注意:}“{\color{red}rmdir}”只能删除{\color{red}空的目录},即该目录下没有其他目录和文件,如果你想把所有目录下的东西都删掉,可以是用命令“rm -r 目录名”,此目录下的所有东西都可删掉。
\subsubsection{cp}
“cp 源文件 目标文件的路径”代表把一个文件复制到另一个文件夹下。

如把test文件夹下的11. 移到桌面上,并重命名为22,“cp test/11.jpg 桌面/22”,用“cd”进入桌面,用“ls”就可以看到桌面出现了一个名为22的图片。上面是在都在家目录下的,如果从用户文件下移到usr里面,用命令“cp test/11.jpg /home/user/22 ”即可。如图\ref{tu33}
\begin{figure}[!htb] %插图
\centering
\includegraphics[width=0.4\textwidth]{tu33.jpeg}
\caption{}
\label{tu33}
\end{figure} 

\subsubsection{rm}
“rm 文件名/目录名”代表移除文件或目录。

{\color{blue}上面所说的“rm -r”一定要先确认以下你是否该目录真的不要了,删除救不不可恢复了,避免误删}。
\subsubsection{mv}
“mv 源文件 目标目录路径”代表移动文件与目录;

“mv 文件名/目录 新的文件名/目录”代表将文件名或者目录名字修改。
\subsubsection{cat}
“cat 文件的路径”代表可以查看这个文件的内容。
\subsubsection{tac}
“tac 文件的路径”代表反向列示文件的内容。与“cat”相比“tac”从最后一行开始在屏幕上显示。
\subsubsection{nl}
“nl 文件的路径”代表添加行号打印。
\subsubsection{od} 
“od -t 文件路径”代表可以查看非纯文本文件的内容
\subsubsection{file}
“file 文件路径”代表可以查看文件的类型。有时候我们对于一个文件的类型不清楚,我们就可以用上面的命令简单的判断一下这个文件的类型了。
\subsubsection{locate}
“locate 部分文件名”代表可以查询出带有这个文件名的文件都会显示出来,但是它不会去硬盘当中访问数据,此时可以用find命令。
\subsubsection{vim程序编辑器}
vim分为三种模式,一般模式、编辑模式、命令行模式,我们一般常用前两种

(1)vim进入一般模式

vim 新文件名或者已有的文件名  (要在文件的目录下执行这步,用vim也可以新建一个文档,但要表明它的类型)

(2)vim进入编辑模式

在一般模式下按“A,a,I,i,O,o,R,r”都可进入编辑模式了,你就可以编写你需要的内容

Esc 退出进入一般模式

如果你想退出编辑,Esc+:wq ,就可以退回到终端
\subsubsection{挂载镜像文件}
sudo mount -o loop /文件路径 /mnt
\subsubsection{安装文件}
sudo apt-get install 软件名(此时软件名是软件源中有的)
\subsubsection{sudo apt-get}
查看一下apt-get的常用命令\ref{tu34}
\begin{figure}[!htb] %插图
\centering
\includegraphics[width=0.4\textwidth]{tu34.jpeg}
\caption{}
\label{tu34}
\end{figure} 
\subsubsection{sudo apt-cache search +文件名}
可以查看这个软件是否在软件源里
\subsubsection{apt-cache depends 软件名}
查询软件可以依赖哪些包
\subsubsection{sudo apt-get autoclean}
 清理旧版本的软件缓存
\subsubsection{sudo apt-get clean}
清理所有软件缓存
\subsubsection{sudo apt autoremove +文件名}
卸载某些文件夹或者文档
\subsubsection{gnome-system-monitor}
关闭一些正在运行的进程,一般用在一个软件卡了的时候。也可以用“killall+程序名”。
\subsection{ubuntu常用快捷}
1、ctrl+shift+T 打开终端

2、以下在终端中用

  (1)ctrl+c 中断终端进程

  (2)ctrl+l 清屏

 (3)ctrl+a 跳到行首

 (4)ctrl+d 从光标向右删除

 (5)Tab  输入首字补全文件名,节省时间

  (6)ctrl+d 在没有输入指令时表示关闭终端

3、ctrl+f 在文档中查询文档关键词 
 
4、ctrl+h 查看隐藏文件

5、alt+	Tab 切换窗口
\end{document}
